\section{Summary and Discussion}
\label{sec:conclusions}

The origin of cosmic magnetism on galactic and extragalactic scales is not fully understood. The competing hypotheses predict a wide range of field strengths, between $10^{-30}$--$10^{-15}$ Gauss comoving for primordial (inflationary and post--inflationary) mechanisms \cite{2013A&ARv..21...62D,2014JCAP...05..040K}, and on the order of $10^{-24}$ Gauss comoving on Mpc scales for Biermann battery \cite{Naoz13} and pre--recombination \cite{2006Sci...311..827I} mechanisms. On the other hand, dynamo mechanism believed to sustain and amplify present--day fields in galaxies typically requires field strengths on the order of $10^{-30}$--$10^{-20}$ Gauss at 10 kpc scales to operate \cite{2001PhLB..501..165D,2002RvMP...74..775W,2006Sci...311..827I}. In this series of papers, we proposed and developed a new method to trace extremely weak cosmic magnetic fields using 21--cm tomography. This method has intrinsic sensitivity to field strengths below $\sim$$10^{-21}$ Gauss comoving at Mpc scales in the IGM prior to cosmic reionization, and could thus start tapping the relevant range of field strengths at high redshift, potentially enabling observational discrimination between various field--origin scenarios with future 21--cm measurements.

In this Paper, we forecast the sensitivity of this method for future 21--cm tomography surveys. For this purpose, we developed a minimum--variance estimator for the magnetic field, which can be applied to measurements of the 21--cm brightness--temperature fluctuations prior to the epoch of reionization. The main numerical results are shown in Figs.~\ref{fig:xi_vs_deltas} and \ref{fig:B_vs_deltas}. They imply that a radio array in a compact--grid configuration with a collecting area slightly larger than one square kilometer can achieve $1\sigma$ sensitivity to a uniform magnetic field of strength $\sim$$10^{-21}$ Gauss comoving, after three years of observation. The case of a stochastic field is more challenging and requires $\sim$10 times larger collecting area to detect a field with a scale--invariant power spectrum. 

Some primordial mechanisms produce fields with power on large scales (see, e.~g.~\cite{2001PhLB..501..165D}), while other mechanisms, in contrast, produce seed fields with blue power spectra (see, e.~g.~\cite{2006Sci...311..827I,Naoz13}). The latter scenario was not directly addressed in this work, but we briefly discuss it now. Namely, in the extreme case where the field has rapid spatial variation and its variance on scales smaller than the wavelength of density fluctuations at hand exceeds the saturation limit, the resulting fast precession of atoms isotropises the incident quadrupole of 21--cm radiation, causing a net reduction of the null--case re--scattered emission quadrupole (rather than transition to another type of quadrupolar emission in the case of a strong field with large coherence length, the case depicted in the bottom panel of Fig.~\ref{fig:hp}). Thus, the presence of such fields with small coherence lengths and large amplitudes can also be traced through their effect on the two--point correlation function of the observed brightness temperature, but the calculation of detection sensitivity presented in this work is not applicable to this case, and is left for future work. On the other hand, if the field has an increasingly larger power on small scales (a blue spectrum), but does not exceed saturation, an estimator analogous to that presented in \S\ref{sec:estimators} for the red--spectrum case can be derived to measure the spectral shape and amplitude. In this case, only modes of the field on large scales would be measurable, while those on small scales would not affect the signal. We also leave a more detailed consideration of this case for future work.

We have only considered an array of dipole antennas in a compact--grid configuration, such as the proposed Fast Fourier Transform Telescope (FFTT) \cite{2009PhRvD..79h3530T}. Our calculations are, however, also applicable to compact arrays of dishes, with the caveat that they have a smaller instantaneous field of view than FFTT and hence have to observe for longer in order to reach the same sensitivity threshold (for a fixed collecting area). Such a design will soon be implemented in the Hydrogen Epoch of Reionization Array (HERA) [29], and our forecasts can be easily rescaled for the next generation of this experiment.

The prospect for measuring cosmological magnetic fields using this method depends on the rate of depolarization of the ground state of hydrogen through Lyman--$\alpha$ pumping, which is proportional to the mean Lyman--$\alpha$ flux prior to reionization. As shown in Fig.~\ref{fig:Bsat}, most of the sensitivity to magnetic fields (for the setup considered in this work) comes from $z\sim 21$, where the Lyman--$\alpha$ flux sufficiently decreases, while the kinetic temperature of the IGM is still low enough. However, the value of the mean Lyman--$\alpha$ flux at these redshifts is completely unconstrained by observation. While the fiducial model we used in our calculations represents one that satisfies modeling constraints and can be extrapolated to match low--redshift observations, it does not capture the full range of possibilities. It is thus important to keep in mind that the projected sensitivity can vary depending on this quantity. We qualitatively capture the variation in projected sensitivity by exploring Lyman--$\alpha$ flux models that vary within a factor of a few from the fiducial model, as shown in Fig.~\ref{fig:cosmo}.

In our analysis, we took into account the noise component arising from Galactic synchrotron emission, but we ignored more subtle effects (such as the frequency dependence of the beams, control of systematic errors from foreground--cleaning residuals, etc.) which may further complicate reconstruction of the magnetic--field signal and should be taken into account when obtaining detailed figures of merit for future experiments.
Finally, we note that the effect of cosmic shear on the 21--cm signal (from weak lensing of the signal by the intervening large scale structure) can produce a noise bias for the magnetic--field measurements. In Appendix \ref{app:lensing}, we examine the level of lensing contamination and show that it is small even for futuristic array sizes of a hundred square kilometers of collecting area. 

It is worth emphasizing again that the main limitation of this method is that it relies on effects that require two--scattering processes. As soon as the quality of cosmological 21--cm statistics reaches the level necessary to probe second--order processes, the effect of magnetic precession we discussed here will lend unprecedented precision to a new \textit{in situ} probe of minuscule, possibly primordial, magnetic fields at high redshifts. 

  