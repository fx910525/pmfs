\section{Fisher analysis}
\label{sec:fisher}

We now use the key results of \S\ref{sec:estimators} to evaluate sensitivity of future tomographic 21--cm surveys to detecting presence of magnetic fields in high---redshift IGM. In \S\ref{subsec:uniform_fisher}, we derive the expression for sensitivity to a field uniform in the entire survey volume. We start with the unsaturated case, and consider the limit where the field (in the classical picture) produces less than 1 radian of precession at all redshifts of interest, and then move on to the saturated (strong field) limit. In \S\ref{subsec:SI_fisher}, we derive the expression for sensitivity to detecting a stochastic magnetic field described by a specific, scale--invariant, power spectrum.

\subsection{Uniform field case}
\label{subsec:uniform_fisher}

Eq.~(\ref{eq:B_estimator_var}) provides an expression for evaluating $1\sigma$ sensitivity to measuring a uniform $B_0$ at a given redshift. The total sensitivity of a tomography survey over a range of redshifts is given by integrating over the available redshift range,
\beq
\bga
\sigma_{ B_0,\text{tot}}^{-2} = 
\frac{1}{2}\int dV_\mathrm{}(z)
\frac{k^2dk d\phi_k\sin \theta_kd\theta_k}{(2\pi)^3}\\
\times\left( \frac{2P_\delta(k,z)G_0(\theta_k,\phi_k,z)\frac{\partial G_0}{\partial B_0}(\theta_k, \phi_k,z)}{P^N(k,\theta_k,z) + P_\delta(k,z)G_0^2(\theta_k,\phi_k,z)} \right)^2,
\ega
\label{eq:fisher_patch}
\eeq
where we transitioned from a sum over $\vec k$ modes to an integral, using $\sum_{\vec k} \to V\int d\vec k /(2\pi)^3$. 
The integral is performed over the (comoving) volume of the survey of angular size $\Omega_\mathrm{survey}$ (in steradians) at a given redshift, such that the volume element reads
\beq
dV_\mathrm{} = \frac{c}{H(z)}\chi^2(z)\Omega_\mathrm{survey}dz.
\label{eq:dVpatch}
\eeq
The integration limits are: $\phi_k\in[0,2\pi]$; $\theta_k\in [0,\pi]$; and $k\in[2\pi u_\mathrm{min}/(d_A\sin\theta_k),2\pi u_\mathrm{max}/(d_A\sin\theta_k)]$, where $u_\mathrm{min, max}=\frac{L_\text{min, max}}{\lambda}$ correspond to the maximum and minimum baseline, $L_\text{min}$ and $L_\text{max}$, respectively. If the survey area is big enough that the flat--sky approximation breaks down, $\sigma_{B_0}^{-2} $ can be computed on small (approximately flat) patch of  size $\Omega_\text{patch}$ and centered on the line of sight, and then corrected to account for the total survey volume
\footnote{This accounts for the change in the angle that a uniform magnetic field makes with a line of sight, as the line of sight ``scans'' through the survey area.}. The corrected sensitivity can be evaluated as
\beq
\bga
\sigma^{-2}_{ B_0,\text{corr}} = \frac{\sigma^{-2}_{ B_0,\text{patch}}}{\Omega_\text{patch}} \int_0^{\theta_\text{survey}}\int_{0}^{2\pi} \cos^2 \theta d\theta d\phi \\
= \frac{\pi\sigma^{-2}_{ B_0,\text{patch}}}{\Omega_\text{patch}} \left(\theta_\text{survey} + \cos \theta_\text{survey} \sin \theta_\text{survey}\right).
\ega
\label{eq:sigma_sum_survey}
\eeq

So far, we have only focused on the regime of the weak magnetic field. Let us now consider the case where the field is strong enough that the precession period is comparable or shorter than the lifetime of the excited atomic state---saturated regime. In this case, the brightness--temperature 2--point correlation functions still capture the presence of the field (as illustrated in Figure \ref{fig:hp}), but they lose information about its magnitude and may only be used to determine the lower limit of the field strength. Ability to distinguish saturated case from zero magnetic field becomes a relevant measure of survey sensitivity in this scenario. 

We now write the signal power spectrum as a sum of contributions from $B_0=0$ and $B_0\to\infty$ scenarios, 
\beq
P^S(\vec k) = (1-\xi)P^S(\vec k, B=0) + \xi P^S(\vec k, B\to \infty),
\label{eq:saturated_P}
\eeq
and perform the standard Fisher analysis to evaluate sensitivity to recovering parameter $\xi$,
\beq
\bga
\sigma_{\xi}^{-2} = 
\int dV_\mathrm{}(z)
\frac{d\vec k}{(2\pi)^3}\left(  \frac{\frac{\partial P^S}{\partial \xi}(\vec k)}{P^N (\vec k)+ P_0^S(\vec k,\xi=0) }\right)^2,
\ega
\label{eq:sigma_xi}
\eeq
where
\beq
\frac{\partial P^S}{\partial \xi}(\vec k) = P^S(\vec k, B\to \infty)-P^S(\vec k, B=0)
\eeq
involves the following limit of the transfer function, derived from Eq.~(\ref{eq:G_def}),
\beq
\bga
G({\bf{\widehat k}}, B\to \infty)
=\left( 1 - \frac{T_\gamma}{T_{\rm s}} \right) x_{1{\rm s}} \left( \frac{1+z}{10} \right)^{1/2} \\
\times \biggl[ 26.4 \ {\rm mK}  \left(1 + ({\bf{\widehat k}} \cdot {\bf{\widehat n}})^2 \right)  
- 0.128 \ {\rm mK} \left( \frac{T_\gamma}{T_{\rm s}} \right)\\
\times x_{1{\rm s}} \left( \frac{1+z}{10} \right)^{1/2}  
 \Bigl\{ 2 + 2({\bf{\widehat k}} \cdot {\bf{\widehat n}})^2 
- \frac{1}{60} \frac{1-3\cos ^2\theta_k}{1+x_{\alpha,(2)}+x_{c,(2)} }\Bigr\} \biggr] ,
\label{eq:G_Binf}
\ega
\eeq
in the reference frame where the magnetic field is along the $z$ axis, and the line-of-sight direction is perpendicular to it; when using this expression to derive numerical results in the following Section, we are only interested in this configuration, since we only evaluate detectability of the components of $\vec B$ in the plane of the sky. We interpret $\sigma_\xi$ as 1$\sigma$ sensitivity to \textit{detecting} presence of a strong magnetic field. 

%%%%%%%%%%%%%%%%%%%%%%%%%%%%%%%%%%%%%%%
\subsection{Stochastic field case}
\label{subsec:SI_fisher}

Using \eq{\ref{eq:NK}} and transitioning from a sum to the integral (like in \S\ref{subsec:uniform_fisher}), we get the following expression for the noise power spectrum of one of the components $B_{0,i}$ of the magnetic field in the plane of the sky,
%\begin{widetext}
\beq
\bga
\left(P^N_{B_{0,i}}(\vec K)\right)^{-1} = \int k^2d{k}\sin \theta_kd\theta_kd\phi_k \\
\times\frac{\left(P_\delta(k')G_0^*({\bf{\widehat k'}})\frac{\partial G_0}{\partial B_i}({\bf{\widehat k'}}) + P_\delta(k)G_0({\bf{\widehat k}})\frac{\partial G_0^*}{\partial B_i}({\bf{\widehat k}})\right)^2}{2(2\pi)^3P_\text{null}(\vec k) P_\text{null}(\vec k') } ,
\ega
\label{eq:NK2}
\eeq
%\end{widetext}
where $\vec k'=\vec K -\vec k$ and the above expression is evaluated at a particular redshift.
To compute signal--to--noise ratio (SNR) for measuring the amplitude of a stochastic--field power spectrum, at a given redshift, we start with the general expression  
\beq
\text{SNR}^2 = \frac{1}{2} \text{Tr} \left( N^{-1}SN^{-1}S\right),
\label{eq:snr_general}
\eeq
where $S$ and $N$ stand for the signal and noise matrices, respectively, and Tr is the trace of the matrix. In our case, these are $3N_\text{voxels}\times 3N_\text{voxels}$ matrices (there are 3 components of the magnetic field and $N_\text{voxels}$ voxels). In the null case, voxels are independent and the noise matrix is diagonal. Voxel--noise  variance for measuring a single mode is given by $P^N_{B_{0,i}}(\vec K, z)/V_\text{voxel} (z)$, where $V_\text{voxel}$ is voxel volume. Summing over all voxels and components of the magnetic field with inverse--variance weights gives
\beq
\bga
\text{SNR}^2 (z)= \frac{1}{2} \sum_{i\alpha, j\beta} \frac{S_{i\alpha , j\beta}^2}{P^N_{B_{0,i}}(\vec K, z)P^N_{B_{0,j}}(\vec K, z)} V_\text{voxel}^2\\=
\frac{1}{2} \sum_{ij} \int d\vec r_\alpha \int d\vec r_\beta \frac{\left< B_{0,i}(\vec r_\alpha) B_{0,j}(\vec r_\beta)\right>^2}{P^N_{B_{0,i}}(\vec K, z)P^N_{B_{0,j}}(\vec K, z)},
\ega
\label{eq:snr_z_step1}
\eeq
at a given redshift. Greek indices label individual voxels and, as before, Roman indices denote field components; $\vec r_{\alpha/\beta}$ represents spatial position of a given voxel. 

To simplify further calculations, we now focus on a particular class of magnetic--field models where most of the power is on largest scales (small $\vec K$). In this (squeezed) limit, $\vec K \ll \vec k$ and thus $\vec k \approx \vec k'$, such that \eq{\ref{eq:NK2}} reduces to the white noise (becomes independent on $\vec K$). A model for the power spectrum is defined through
\beq
(2\pi)^3\delta_D(\vec K - \vec K') P_{B_{0,i}B_{0,j}}(\vec K) \equiv \left<B_{0,i}^*(\vec K) B_{0,j}(\vec K')\right>,
\label{eq:Pbb}
\eeq
which relates to the variance in the transverse component $P_{B_\bot}(\vec K)$ as
\beq
P_{B_{0,i}B_{0,j}}(\vec K) = (\delta_{ij} - \widehat K_i \widehat K_j) P_{B_\bot}(\vec K),
\label{eq:Pbb_Pb}
\eeq
where $\widehat K_{i/j}$ is a unit vector along the direction of ${i/j}$ component.
In the rest of this discussion, for concreteness, we consider a scale--invariant (SI) power spectrum, 
\beq
P_{{B_\bot}}(\vec K) = A_0^2/K^3.
\label{eq:SI}
\eeq
Here, the amplitude $A_0$ is a free parameter of the model (in units of Gauss). Furthermore, if homogeneity and isotropy are satisfied, the integrand in Eq.~(\ref{eq:snr_z_step1}) only depends on the separation vector $\vec s \equiv \vec r_\beta -\vec r_\alpha$. Using this, and the squeezed limit assumption, gives\footnote{In the last step, we used $\int d\vec s |f(\vec s)|^2 = \int \frac{d\vec K}{(2\pi)^3}|\widetilde f(\vec K)|^2$, which holds for an arbitrary function $f$ and its Fourier transform $\widetilde f$.}
\beq  
\bga
\text{SNR}^2 (z) = 
\frac{1}{2} \sum_{ij}  \frac{dV_\text{patch}}{{(P^N_{B_{0,i}}(z))^2}}\int d\vec s \left< B_{0,}i(\vec r_\beta - \vec s) B_{0,j}(\vec r_\beta)\right>^2
\\=
\frac{1}{2(2\pi)^3} \sum_{ij}  \frac{dV_\text{patch}}{{(P^N_{B_{0,i}}(z))^2}} \int d\vec K\left(P_{B_{0,i}B_{0,j}}(\vec K)\right)^2,
\ega
\label{eq:snr_z}
\eeq
where $dV_\text{patch}$ is the volume of a redshift--slice patch defined in \eq{\ref{eq:dVpatch}}. Substituting \eq{\ref{eq:SI}}, and integrating over redshifts, total SNR is given by
\beq
\bga
\text{SNR}^2 =  \frac{A_0^4}{2(2\pi)^3}  \int_{z_\text{min}}^{z_\text{max}}\frac{dV_\text{patch}}{{(P^N_{B_{0,i}}(z))^2}}
\int_0^{\pi} \sin\theta d\theta \\
\int_0^{2\pi} d\phi\int_{K_\text{min}(z,\theta,\phi)}^{K_\text{max}(z,\theta,\phi)} \frac{d K}{K^4}\sum_{ij\in \{xx, xy, yx, yy\}}(\delta_{ij} - \widehat K_i\widehat K_j)^2,
\ega
\label{eq:snr_intK}
\eeq
where $x$ and $y$ denote components in the plane of the sky, and
\beq
\widehat K_x = \sin\theta\sin\phi, \text{     }
\widehat K_y = \sin\theta\cos\phi.
\label{eq:hat_K_xy}
\eeq
The sum in the above expression reduces to
\beq
\sum_{ij\in \{xx, xy, yx, yy\}}(\delta_{ij} - \widehat K_i\widehat K_j)^2 = 2\cos^2\theta+\sin^4\theta.
\label{eq:sumij}
\eeq
Substituting this into Eq.~(\ref{eq:snr_intK}) and integrating over $K,\theta,\phi$ gives
\beq
\text{SNR}^2 =  \frac{A_0^4}{10\pi^2} \int_{z_\text{min}}^{z_\text{max}}\frac{dV_\text{patch}}{{(P^N_{B_{0,i}}(z))^2}} \left(\frac{1}{K_\text{min}^3}-\frac{1}{K_\text{max}^3}\right).
\label{eq:snr_ints}
\eeq
Finally, from the above expression, $1\sigma$ sensitivity to measuring $A_0^2$ is given by
\beq
\sigma^2_{A_0^2} =  \left[\frac{1}{10\pi^2} \int_{z_\text{min}}^{z_\text{max}}\frac{dV_\text{patch}}{{(P^N_{B_{0,i}}(z))^2}} \left(\frac{1}{K_\text{min}^3}-\frac{1}{K_\text{max}^3}\right)\right]^{-1}.
\label{eq:sigma_A0}
\eeq
Note at the end that, for our choice of the SI power spectrum, the choice of $K_\text{max}$ does not matter, while we choose $K_\text{min}$ to match the survey size at a given redshift. 