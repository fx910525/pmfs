\section{Fisher analysis}
\label{sec:fisher}

We now use the key results of \S\ref{sec:estimators} to evaluate expressions for sensitivity of future observations. We first derive the expression for sensitivity to a field uniform in the entire survey volume. We start with the unsatured case, and consider the limit where the field (in the classical picture) produces less than 1 radian of precession at all redshifts of interest, and then move on to the saturated (strong field) limit. Secondly, we derive the expression for sensitivity to detecting a stochastic magnetic field with a scale-independent power spectrum.

\subsection{Uniform field case}
\label{subsec:uniform_fisher}

For a measurement of the redshifted 21-cm brightness temperature signal at a given z, the sensitivity $\sigma_{\widehat B_0}$ to recovering a uniform field $B_0$ in unsaturated limit is given by Eq.~(\ref{eq:B_estimator_var}). The total sensitivity of a tomorgraphy survey over a range of redshifts is given by
\beq
\bga
\sigma_{B_0}^{-2} = 
\int dV_\mathrm{}(z)
\frac{k^2dk d\phi_k\sin \theta_kd\theta_k}{(2\pi)^3}\\
\times\left( \frac{2P_\delta(k,z)G_0(\theta_k,\phi_k,z)\frac{\partial G_0}{\partial B_0}(\theta_k, \phi_k,z)}{P^N(k,\theta_k,z) + P_\delta(k,z)G_0^2(\theta_k,\phi_k,z)} \right)^2,
\ega
\label{eq:fisher_patch}
\eeq
where we transitioned from a sum over $\vec k$ modes to an integral, using $\sum_{\vec k} \to V\int d\vec k /(2\pi)^3$. 
The integral is performed over the (comoving) volume of the survey of angular size $\Omega_\mathrm{survey}$ (in steradians) at a given redshift, 
\beq
dV_\mathrm{} = \frac{c}{H(z)}\chi^2(z)\Omega_\mathrm{survey}dz,
\label{eq:dVpatch}
\eeq
and the integration limits are: $\phi_k\in[0,2\pi]$; $\theta_k\in [0,\pi]$; and $k\in[2\pi u_\mathrm{min}/(d_A\sin\theta_k),2\pi u_\mathrm{max}/(d_A\sin\theta_k)]$, where $u_\mathrm{min, max}=\frac{L_\text{min, max}}{\lambda}$ correspond to the maximum and minimum baseline $L_\text{min}$ and $L_\text{max}$, respectively. If the survey area is big enough that the flat-sky approximation breaks down, $\sigma_{B_0}^{-2} $ can be computed on small (approximately flat) patch $\Omega_\text{patch}$, and corrected in the following way to account for the total survey volume
\footnote{This accounts for the change in the angle that a uniform magnetic field makes with a line of sight, as the line of sight moves through a large survey area.}
\beq
\bga
\sigma^{-2}_{\widehat B_0,\text{ tot}} = \frac{\sigma^{-2}_{\widehat B_0}}{\Omega_\text{patch}} \int_0^{\theta_\text{survey}}\int_{0}^{2\pi} \cos^2 \theta d\theta d\phi \\
= \frac{\sigma^{-2}_{\widehat B_0}\pi}{\Omega_\text{patch}} \left(\theta_\text{survey} + \cos \theta_\text{survey} \sin \theta_\text{survey}\right).
\ega
\label{eq:sigma_sum_survey}
\eeq

%%%   MORE HERE!!!
Let us now consider a field that is strong enough to produce precession by more than a radian in a lifetime of the excited state of the 21-cm transition; this is the saturated-signal case. The brightness-temperature 2-point correlation functions still capture the presence of the field in this case (as illustrated in Figure \ref{fig:hp}), but it loses sensitivity to recovering its exact magnitude. Ability to distinguish staturated case from zero magnetic field becomes a relevant measure of sensitivity in this case. 

To evaluate the sensitivity, we can write the signal power spectrum as a sum of contributions from $B_0=0$ and $B_0\to\infty$ cases, 
\beq
P^S(\vec k) = (1-\xi)P^S(\vec k, B=0) + \xi P^S(\vec k, B\to \infty),
\label{eq:saturated_P}
\eeq
and perform the standard Fisher analysis to evaluate sensitivity to recovering parameter $\xi$,
\beq
\bga
\sigma_{\xi}^{-2} = 
\int dV_\mathrm{}(z)
\frac{d\vec k}{(2\pi)^3}\left(  \frac{\frac{\partial P^S}{\partial \xi}(\vec k)}{P^N (\vec k)+ P_0^S(\vec k,\xi=0) }\right)^2. 
\ega
\label{eq:sigma_xi}
\eeq
In this case, we interpret $\sigma_\xi$ as a 1$\sigma$ sensitivity to \textit{detecting} presence of a strong magnetic field.

%%%%%%%%%%%%%%%%%%%%%%%%%%%%%%%%%%%%%%%
\subsection{Stochastic field case}
\label{subsec:stochastic_fisher}

Using \eq{\ref{eq:NK}} and a procedure analogous to the case of a uniform field, we get the following integral expression for the noise power spectrum of a plane-of-the-sky component of the magnetic field 
%\begin{widetext}
\beq
\bga
\left(P^N_{B_{0,i}}(\vec K)\right)^{-1} = \int k^2d{k}\sin \theta_kd\theta_kd\phi_k \\
\times\frac{\left(P_\delta(k')G_0^*({\bf{\widehat k'}})\frac{\partial G_0}{\partial B_i}({\bf{\widehat k'}}) + P_\delta(k)G_0({\bf{\widehat k}})\frac{\partial G_0^*}{\partial B_i}({\bf{\widehat k}})\right)^2}{2(2\pi)^3P_\text{null}(\vec k) P_\text{null}(\vec k') } ,
\ega
\label{eq:NK2}
\eeq
%\end{widetext}
where $\vec k'=\vec K -\vec k$.

To compute signal-to-noise ratio (SNR) for measuring the amplitude of an arbitrary stochastic-field power spectrum in a given redshift slice $z$ we need to perform a sum over all voxels (3d pixels) in the survey volume at that redshift. A general expression for SNR is 
\beq
\text{SNR}^2 = \frac{1}{2} Tr \left( N^{-1}SN^{-1}S\right),
\label{eq:snr_general}
\eeq
where S and N are the signal and noise matrices. In our case, these are $3N_\text{voxels}\times 3N_\text{voxels}$ matrices (there are $N_\text{voxels}$ voxels in the entire survey, and 3 components of the magnetic field). In the null case, voxels are independent, and so the noise matrix is diagonal, and the signal is captured by the 3d power spectrum of the magnetic field. The voxel-noise  variance for measuring a single mode is given by $P^N_{B_{0,i}}(\vec K, z)/V_\text{voxel} (z)$, where $V_\text{voxel}$ is volume of a given voxel. Summing over voxels and components of the magnetic field with inverse-variance weights, for a single redshift slice, we get SNR as
\beq
\bga
\text{SNR}^2 (z)= \frac{1}{2} \sum_{i\alpha, j\beta} \frac{S_{i\alpha , j\beta}^2}{P^N_{B_{0,i}}(\vec K, z)P^N_{B_{0,j}}(\vec K, z)} V_\text{voxel}^2\\=
\frac{1}{2} \sum_{ij} \int d\vec r_\alpha \int d\vec r_\beta \frac{\left< B_{0,i}(\vec r_\alpha) B_{0,j}(\vec r_\beta)\right>^2}{P^N_{B_{0,i}}(\vec K, z)P^N_{B_{0,j}}(\vec K, z)},
\ega
\label{eq:snr_z_step1}
\eeq
where we label voxels with Greek indicies, and, as before, retain Roman indicies for the field components; $\vec r_{\alpha/\beta}$ represents the spatial position of a given voxel. 

To simplify further calculations, we now only focus on a particular class of magnetic-field models, where most of the power is on largest scales (small $\vec K$). In this (squeezed) limit, $\vec K \ll \vec k$ and thus $\vec k \approx \vec k'$, such that \eq{\ref{eq:NK2}} reduces to the white noise (independent on $\vec K$). The model for the power spectrum is defined through
\beq
(2\pi)^3\delta_D(\vec K - \vec K') P_{B_{0,i}B_{0,j}}(\vec K) = \left<B_{0,i}^*(\vec K) B_{0,j}(\vec K')\right>,
\label{eq:Pbb}
\eeq
which relates to the variance in the transverse component $P_{B_\bot}(\vec K)$ as
\beq
P_{B_{0,i}B_{0,j}}(\vec K) = (\delta_{ij} - \widehat K_i \widehat K_j) P_{B_\bot}(\vec K),
\label{eq:Pbb_Pb}
\eeq
where $\widehat K_{i/j}$ is a unit vector along the direction of ${i/j}$ component.
In this discussion, as a concrete model example, we consider a scale--independent (SI) power spectrum, 
\beq
P_{{B_\bot}}(\vec K) = A_0^2/K^3,
\label{eq:SI}
\eeq
where the amplitude $A_0$ is a free parameter (in units of Gauss). Furthermore, if homogeneity and isotropy are satisfied, the integrand in Eq.~(\ref{eq:snr_z_step1}) only depends on the separation vector $\vec s \equiv \vec r_\beta -\vec r_\alpha$. Using this and the squeezed limit assumption gives\footnote{Note that in the last step we used $\int d\vec s |f(\vec s)|^2 = \int \frac{d\vec K}{(2\pi)^3}|\widetilde f(\vec K)|^2$, which holds for an arbitrary function $f$ and its Fourier transform $\widetilde f$.}
\beq  
\bga
\text{SNR}^2 (z) = 
\frac{1}{2} \sum_{ij}  \frac{dV_\text{patch}}{{(P^N_{B_{0,i}}(z))^2}}\int d\vec s \left< B_{0,}i(\vec r_\beta - \vec s) B_{0,j}(\vec r_\beta)\right>^2
\\=
\frac{1}{2(2\pi)^3} \sum_{ij}  \frac{dV_\text{patch}}{{(P^N_{B_{0,i}}(z))^2}} \int d\vec K\left(P_{B_{0,i}B_{0,j}}(\vec K)\right)^2,
\ega
\label{eq:snr_z}
\eeq
where $dV_\text{patch}$ is the volume of a redshift--slice patch of \eq{\ref{eq:dVpatch}}. Substituting \eq{\ref{eq:SI}}, and integrating over all $z$'s available in the survey, total SNR reads
\beq
\bga
\text{SNR}^2 =  \frac{A_0^4}{2(2\pi)^3}  \int_{z_\text{min}}^{z_\text{max}}\frac{dV_\text{patch}}{{(P^N_{B_{0,i}}(z))^2}}
\int_0^{\pi} \sin\theta d\theta \\
\int_0^{2\pi} d\phi\int_{K_\text{min}(z,\theta,\phi)}^{K_\text{max}(z,\theta,\phi)} \frac{d K}{K^4}\sum_{ij\in \{xx, xy, yx, yy\}}(\delta_{ij} - \widehat K_i\widehat K_j)^2,
\ega
\label{eq:snr_intK}
\eeq
where $x$ and $y$ denote components in the plane of the sky, and
\beq
\widehat K_x = \sin\theta\sin\phi, \text{     }
\widehat K_y = \sin\theta\cos\phi.
\label{eq:hat_K_xy}
\eeq
The sum in the above expression reduces to
\beq
\sum_{ij\in \{xx, xy, yx, yy\}}(\delta_{ij} - \widehat K_i\widehat K_j)^2 = 2\cos^2\theta+\sin^4\theta.
\label{eq:sumij}
\eeq
Substituting this into \eq{\ref{eq:snr_intK}}, and integrating over $K,\theta,\phi$ gives
\beq
\text{SNR}^2 =  \frac{A_0^4}{10\pi^2} \int_{z_\text{min}}^{z_\text{max}}\frac{dV_\text{patch}}{{(P^N_{B_{0,i}}(z))^2}} \left(\frac{1}{K_\text{min}^3}-\frac{1}{K_\text{max}^3}\right).
\label{eq:snr_ints}
\eeq
From this expression, $1\sigma$ sensitivity to measuring $A_0^2$ is
\beq
\sigma_{A_0^2} =  \left[\frac{1}{10\pi^2} \int_{z_\text{min}}^{z_\text{max}}\frac{dV_\text{patch}}{{(P^N_{B_{0,i}}(z))^2}} \left(\frac{1}{K_\text{min}^3}-\frac{1}{K_\text{max}^3}\right)\right]^{-\frac{1}{2}}.
\label{eq:sigma_A0}
\eeq