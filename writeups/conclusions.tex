\section{Summary and Conclusions}
\label{sec:conclusions}
%mention lensing here

In Paper I of this series, we proposed a new method to detect extremely weak magnetic fields in the IGM during the Dark Ages, using 21--cm tomography. In this paper, Paper II, we investigated sensitivity of this method with future 21--cm tomography surveys. For this purpose, we developed a minimum--variance--estimator formalism that uses measurements of the 21--cm brightness temperature to probe magnetic fields prior to the epoch of reionization. 

The main results are in Figures \ref{fig:xi_vs_deltas} and \ref{fig:B_vs_deltas}. Their implication is that a radio array in a compact--grid configuration with a square kilometer collecting area has the sensitivity necessary to put an upper bound of $10^{-21}$ Gauss comoving on a uniform magnetic fields at high redshifts. The case of a stochastic field is more challenging (by a factor of a few in the case of a scale--invariant power spectrum), and measuring the spectral shape of such a field would require even larger arrays to achieve. In this analysis, we took into account the noise component arising from the presence of large Galactic foreground signal, but we ignored more other effects (such as, for example, frequency dependence of the beams, etc.) which may further complicate reconstruction of such signal and should be taken into account in a detailed analyses for figures of merit for future experiments. Finally, we note that gravitational lensing of the 21--cm signal by the intervening large scale structure can in principle present a contaminant for the magnetic field measurement. In Appendix \ref{app:lensing}, we examine this possibility and show that the contamination is negligible for arrays with coverage areas considered in this work.

A kilometer squared of collecting area corresponds to a radio--array size planned for the next stages of some of the current reionization--epoch experiments (in terms of the number of antennas, compare to HERA and to the SKA \cite{2008arXiv0802.1727C}, for example). The number of mode measurements required for placing a meaningful upper limit on such early--time magnetic fields using our method does not supersede computational demands for the next--generation  experiments and is thus achievable in the coming future. Finally, it is also worth emphasizing again that the main limitation to the sensitivity is the fact that the effects we considered are based on a two--scattering process---as soon as quality of the 21--cm statistics reaches the level necessary to probe second--order processes, the effect of magnetic precession we discussed in this series of papers will open up an ``\textit{in situ}'' probe of minuscule (and possibly primordial) magnetic fields at high redshift with unprecedented precision. 

  