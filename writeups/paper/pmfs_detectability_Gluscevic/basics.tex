\section{Basics}
\label{sec:basics}

Before focusing on the estimator formalism (presented in the following Section), we review the basics of 21--cm brightness--temperature fluctuation measurements. In \S\ref{subsec:def}, we set up our notation and review definitions of quantities describing sensitivity of interferometric radio arrays; in \S\ref{subsec:noise}, we focus on the derivation of the noise power spectrum; and in \S\ref{subsec:uv}, we discuss the effects of the array configuration and its relation to coverage of modes in the $uv$ plane. 

\subsection{Definitions}
\label{subsec:def}

The redshifted 21--cm signal can be represented with specific intensity at a location in physical space $I(\vec{r})$ or in Fourier space $\widetilde{I}(\vec{k})$. In sky coordinates (centered on an emitting patch of the sky), these functions become $\mathcal{I}(\theta_x, \theta_y, \theta_\nu)$ and $\widetilde{\mathcal{I}}(u,v,\eta)$, respectively. Here, vector $\vec{k}$ (in the units of comoving Mpc$^{-1}$) is a Fourier dual of $\vec{r}$ (comoving Mpc), and likewise, $\theta_x$ (rad), $\theta_y$ (rad), and $\theta_\nu$ (Hz) are duals of the coordinates $u$ (rad$^{-1}$), $v$ (rad$^{-1}$), and $\eta$ (seconds), respectively.  Notice that $\theta_x$ and $\theta_y$ represent the angular extent of the patch in the sky, while $\theta_\nu$ represents its extent in frequency space. The two sets of coordinates are related through linear transformations in the following way
\begin{equation}
\begin{gathered}
\theta_x = \frac{r_x}{\chi(z)}, \hspace{0.5in} u = \frac{k_x\chi(z)}{2\pi},\\
\theta_y = \frac{r_y}{\chi(z)}, \hspace{0.5in} v = \frac{k_y\chi(z)}{2\pi},\\
\theta_\nu = \frac{H(z)\nu_{21}}{c(1+z)^2} r_z, \hspace{0.5in} \eta = \frac{c(1+z)^2}{2\pi H(z)\nu_{21}}k_z,
\end{gathered}
\label{eq:fourier_duals}
\end{equation} 
where $\nu_{21}=1420.4$ MHz is the frequency corresponding to the 21--cm line in the rest frame of the emitting atoms; $H(z)$ is the Hubble parameter; and $\chi(z)$ is the comoving distance to redshift $z$ which marks the middle of the observed data cube where $r_z$ and $\theta_\nu$ intervals are evaluated. Note that $2\pi\theta_iu = r_ik_i$, for $i\in\{x,y\}$. The convention we use for the Fourier transform is 
\beq
\bga
I(\vec{r}) = \frac{1}{(2\pi)^3}\int\widetilde{I}(\vec{k})e^{i\vec{k} \cdot \vec{r}}d\vec{k},\\
\widetilde{I}(\vec{k}) = \int{I}(\vec{r})e^{-i\vec{k} \cdot \vec{r}}d\vec{r},
\ega
\label{eq:tildeI_I}
\eeq
where Fourier--space functions are denoted with tilde. Similarly,
\beq
\bga
\mathcal{I}(\theta_x,\theta_y,\theta_\nu) = \int\widetilde{\mathcal{I}}(u,v,\eta)e^{2\pi i(u\theta_x + v\theta_y+\eta \theta_\nu)}dudvd\eta,\\
\widetilde{\mathcal{I}}(u,v,\eta) = \int{\mathcal{I}}(\theta_x,\theta_y,\theta_\nu)e^{-2\pi i(u\theta_x + v\theta_y+\eta\theta_\nu)}d\theta_xd\theta_yd\theta_\nu.
\ega
\label{eq:mathcal_tilde_I}
\eeq
From Eqs.~(\ref{eq:fourier_duals})--(\ref{eq:mathcal_tilde_I}), the following relation is satisfied
\beq
\widetilde{I}(\vec{k}) = \frac{c(1+z)^2\chi(z)^2}{H(z)\nu_{21}}\widetilde{\mathcal{I}}(u,v,\eta),
\label{eq_tilde_I_vs_Ik_scaling}
\eeq
where the proportionality factor contains the transformation Jacobian $\frac{dr_xdr_ydr_z}{d\theta_xd\theta_yd\theta_\nu}$. Finally, the relationship between the specific intensity in the $uv$--plane and the visibility function $\mathcal V(u,v,\theta_\nu)$ is given by the Fourier transform of the frequency coordinate,
\beq
\bga
 \mathcal{V}(u,v,\theta_\nu)= \int \mathcal{\widetilde{I}}(u,v,\eta)e^{2\pi i \theta_\nu\eta}d\eta,\\
\mathcal{\widetilde{I}}(u,v,\eta) = \int \mathcal{V}(u,v,\theta_\nu)e^{-2\pi i \theta_\nu\eta}d\theta_\nu.
\ega
\label{eq:visibility}
\eeq
Here, $\theta_{\nu,\text{max}}-\theta_{\nu,\text{min}}=\Delta\nu$ is the bandwidth of the observed data cube, centered on $z$ (see also Appendix \ref{app:Vrms}).
%%%%%%%%%%%%%%
\subsection{Power spectra and noise}
\label{subsec:noise}

In this Section, we derive the noise power spectrum for the brightness--temperature fluctuation measurement. We start by defining a brightness--temperature power spectrum as
\beq
\langle \widetilde{I}(\vec{k})\widetilde{I}^*(\vec{k}')\rangle \equiv (2\pi)^3P_{\widetilde{I}}\delta_D(\vec{k}-\vec{k}'),
\label{eq_tildeI_power}
\eeq
where $\delta_D$ is Dirac delta function. The observable quantity of the interferometric arrays is the visibility function---a complex Gaussian variable with a zero mean and the following variance (see detailed derivation in Appendix \ref{app:Vrms}) 
\beq\bga
\langle \mathcal{V}({u},v,\theta_\nu)\mathcal{V}({u'},v',\theta_\nu')^*\rangle \\
= \frac{1}{\Omega_\text{beam}}\left(\frac{2k_BT_\text{sky}}{A_e\sqrt{\Delta\nu t_1}}\right)^2 \delta_D({u}-{u}')\delta_D({v}-{v}')\delta_{\theta_\nu\theta_{\nu}'},
\ega
\label{eq_Vrms}
\eeq 
where $T_\text{sky}$ is the sky temperature (which, in principle, includes both the foreground signal from the Galaxy, and the instrument noise, where we assume the latter to be subdominant in the following); $t_1$ is the total time a single baseline observes element $(u,v)$ in the $uv$ plane; $A_e$ is the collecting area of a single dish; $k_B$ is the Boltzmann constant; $\Delta\nu$ is the bandwidth of a single observation centered on $z$; and the last $\delta$ in this expression denotes the Kronecker delta.

Combining Eqs.~(\ref{eq:visibility}) and (\ref{eq_Vrms}), and taking the ensemble average,
\beq
\bga
\langle\widetilde{\mathcal{I}}(u,v,\eta) \widetilde{\mathcal{I}}^*(u',v',\eta')\rangle \\
 = \frac{1}{t_1\Omega_\text{beam}}\left(\frac{2k_BT_\text{sky}}{A_e}\right)^2 \delta_D({u}-{u}')\delta_D({v}-{v}')\delta_D(\eta-\eta'),
\ega
\label{eq:mathcal_power_Vrms}
\eeq 
where we used the standard definition
\beq
\int e^{2\pi i \theta_\nu(\eta-\eta')}d\theta_\nu =\delta_D(\eta-\eta').
\eeq
Taking into account the relation of \eq{\ref{eq_tilde_I_vs_Ik_scaling}}, using \eq{\ref{eq_tildeI_power}}, and keeping in mind the scaling property of the delta function, we arrive at
\beq
P_1^N(\vec k) = \frac{c(1+z)^2\chi^2(z)}{\Omega_\text{beam}t_1H(z)\nu_{21}}\left(\frac{2k_BT_\text{sky}}{A_e}\right)^2 ,
\label{eq:Pnoise_1mode}
\eeq
for the noise power per $\vec k$ mode, per baseline.

In the last step, we wish to get from \eq{\ref{eq:Pnoise_1mode}} to the expression for the noise power spectrum that corresponds to observation with all available baselines. To do that, we need to incorporate information about the array configuration and its coverage of the $uv$ plane. In other words, we need to divide the expression in \eq{\ref{eq:Pnoise_1mode}} by the number density of baselines $n_\text{base}(\vec k)$ that observe a given mode $\vec k$ at a given time (for a discussion of the $uv$ coverage, see the following Section). The final result for the noise power spectrum per mode $\vec k$ in intensity units is 
\beq
P^N(\vec k) = \frac{c(1+z)^2\chi^2(z)}{\Omega_\text{beam}t_1H(z)\nu_{21}}\frac{\left(2k_BT_\text{sky}\right)^2}{A_e^2n_\text{base}(\vec k)},
\label{eq:Pnoise_Jy}
\eeq
and in temperature units
\beq
P^N(\vec k) =\frac{\lambda^4c(1+z)^2\chi^2(z)}{\Omega_\text{beam}t_1H(z)\nu_{21}}\frac{T_\text{sky}^2}{A_e^2n_\text{base}(\vec k)},
\label{eq:Pnoise_K}
\eeq
where $\lambda=c/\nu_{21}(1+z)$.

%%%%%%%%%%%%
\subsection{The UV coverage}
\label{subsec:uv}

The total number density $n_\text{base}(\vec k)$ of baselines that can observe mode $\vec k$ is related to the (unitless) number density $n(u,v)$ of baselines per $dudv$ element as
\beq
n_\text{base}(\vec k) = \frac{n(u,v)}{\Omega_\text{beam}},
\label{eq:nuv_nk}
\eeq
where $\frac{1}{\Omega_\text{beam}}$ represents an element in the $uv$ plane. The number density integrates to the total number of baselines $N_\text{base}$,
\beq
N_\text{base}=\frac{1}{2}N_\text{ant}(N_\text{ant}+1) = \int_\text{half} n(u,v)dudv,
\label{eq:nk}
\eeq
where $N_\text{ant}$ is the number of antennas in the array, and the integration is done on one half of the $uv$ plane\footnote{This is because the visibility has the following property $V(u,v,\theta_\nu)=V^*(-u,-v,\theta_\nu)$, and only a half of the plane contains independent samples.}. We assume that the array consists of many antennas, so that time dependence of $n(u,v)$ is negligible; if this is not the case, time average of this quantity should be computed to account for Earth's rotation.

In this work, we focus on a specific array configuration that is of particular interest to cosmology---a compact grid of dipole antennas, with a total collecting area of $(\Delta L)^2$, and a maximum baseline length\footnote{Note that for a square with area $(\Delta L)^2$ tiled in dipoles, there is a very small number of baselines longer than $\Delta L$, but we neglect this for simplicity.} of $\Delta L$. In this setup, the beam solid angle is 1 sr, the effective area of a single dipole is $A_e = \lambda^2$, and the effective number of antennas is $N_\text{ant} = \frac{(\Delta L)^2}{\lambda^2}$. For such a configuration, the number density of baselines entering the calculation of the noise power spectrum reads
\beq
n(u,v) = (\frac{\Delta L}{\lambda} - u)(\frac{\Delta L}{\lambda} - v).
\label{eq:nuv_fftt}
\eeq
The relation between $\vec k=(k,\theta_k,\phi_k)$ and $(u,v)$ is
\beq
\bga
u_\perp \equiv \frac{\chi(z)}{2\pi}k\sin\theta_k,\\
u = u_\perp \cos\phi_k,\\
v = u_\perp \sin\phi_k,
\ega
\label{eq:k_uv}
\eeq
where the subscript $\perp$ denotes components perpendicular to the line--of--sight direction ${\bf{\widehat n}}$, which, in this case, is along the $z$ axis. From this, the corresponding number of baselines observing a given $\vec k$ is
\beq
\bga
n_\text{base}(\vec k) = (\frac{\Delta L}{\lambda} - \frac{\chi(z)}{2\pi}k\sin\theta_k\cos\phi_k)\\\times (\frac{\Delta L}{\lambda} - \frac{\chi(z)}{2\pi}k\sin\theta_k\sin\phi_k).
\ega
\label{eq:nk_fftt}
\eeq

As a last note, when computing numerical results in \S\ref{sec:results}, we substitute the $\phi_k$--averaged version of the above quantity (averaged between $0$ and $\pi/2$ only, due to the four--fold symmetry of the experimental setup of a square of dipoles) when computing the noise power, in order to account for the rotation of the baselines with respect to the modes in the sky. This average number density reads
\beq
\bga
\langle n_\text{base}(\vec k) \rangle_{\phi_k}= \left(\frac{\Delta L}{\lambda}\right)^2 -\frac{4}{\pi}\frac{\Delta L}{\lambda}\frac{\chi(z)}{2\pi}k\sin\theta_k \\+ \frac{1}{\pi}\left(\frac{\chi(z)}{2\pi}k\sin\theta_k\right)^2,
\ega
\label{eq:nk_fftt_mean}
\eeq
assuming a given mode $k$ is observable by the array, such that its value is between $2\pi L_\text{min}/(\lambda(z)\chi(z)\sin\theta_k)$ and $2\pi L_\text{max}/(\lambda(z)\chi(z)\sin\theta_k)$, where $L_\text{min}$ and $L_\text{max}$ are the maximum and minimum baseline lengths, respectively. If this condition is not satisfied, $\langle n_\text{base}(\vec k) \rangle_{\phi_k}=0$.