\section{Summary and Discussion}
\label{sec:conclusions}

In Paper I of this series, we proposed a new method to detect extremely weak magnetic fields in the IGM during the cosmic Dark Ages, using 21--cm tomography. In this Paper, we forecast the sensitivity of our method considering the upcoming and future 21--cm tomography surveys. For this purpose, we developed a minimum--variance estimator for the magnetic field, which can be applied to the measurements of the 21--cm brightness--temperature fluctuations prior to the epoch of reionization. While we only considered an array of dipole antennas in a compact--grid configuration, a similar configuration has been proposed for the Fast Fourier Transform Telescope (FFTT) \cite{2009PhRvD..79h3530T} and is also being implemented for an array of dishes in Hydrogen Epoch of Reionization Array (HERA) \cite{2015AAS...22532803D}, for example. Our results thus have most direct implications for those and similar experimental setups.

The main results are shown in Figs.~\ref{fig:xi_vs_deltas} and \ref{fig:B_vs_deltas}. They imply that a radio array in a compact--grid configuration with a collecting area slightly larger than one square kilometer can achieve $1\sigma$ sensitivity to a uniform magnetic field of strength $\sim10^{-21}$ Gauss comoving, after three years of observation. The case of a stochastic field is more challenging (by a factor of a few in the case of a field with a scale--invariant power spectrum), and detection in that case would require $\sim 10$ times larger collecting areas.

The prospect for measuring magnetic fields using this method depends on the rate of depolarization of the ground state of hydrogen through Lyman--$\alpha$ pumping, which is proportional to the Lyman--$\alpha$ flux prior to reionization. As shown in Fig.~\ref{fig:Bsat}, most of the sensitivity to magnetic fields comes from $z\sim 21$, where the Lyman--$\alpha$ flux sufficiently decreases, while the kinetic temperature of the IGM is still low enough. However, the value of the Lyman--$\alpha$ flux at these redshifts is completely unconstrained by observation. While the fiducial model we used in our calculations represents one that satisfies modeling constraints and can be extrapolated to match low--redshift observations, it does not capture the full range of possibilities. It is thus important to keep in mind that the projected sensitivity can vary depending on this quantity. We qualitatively capture the variation in projected sensitivity by exploring Lyman--$\alpha$ flux models that stay within a factor of a few from the fiducial model, as shown in Fig.~\ref{fig:cosmo}.

In our analysis, we took into account the noise component arising from Galactic synchrotron emission, but we ignored more subtle effects (such as the frequency dependence of the beams, control of systematic errors from foreground--cleaning residuals, etc.) which may further complicate reconstruction of the magnetic--field signal and should be taken into account when obtaining figures of merit for future experiments.
Finally, we note that the effect of cosmic shear on the 21--cm signal (from weak lensing of the signal by the intervening large scale structure) can produce a noise bias for the magnetic--field measurements. In Appendix \ref{app:lensing}, we examine the level of lensing contamination and show that it is negligible even for futuristic array sizes of a hundred square kilometers of collecting area. 

An array with one square kilometer of collecting area corresponds to the plans for the next stages of some of the current reionization--epoch experiments (in terms of the number of antennas, compare to HERA and to the SKA \cite{2008arXiv0802.1727C}, for example). The number of mode measurements required for placing a meaningful upper limit on the early--time magnetic fields with the method proposed in this work does not supersede computational demands for the next--generation  experiments, and is thus achievable in the coming future. It is worth emphasizing again that the main limitation of this method is that it relies on effects that require two--scattering processes. As soon as the quality of cosmological 21--cm statistics reaches the level necessary to probe second--order processes, the effect of magnetic precession we discussed here will lend unprecedented precision to a new \textit{in situ} probe of minuscule, possibly primordial, magnetic fields at high redshifts. 

  