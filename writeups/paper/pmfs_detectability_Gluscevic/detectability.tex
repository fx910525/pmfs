\documentclass[aps,prd,twocolumn,floatfix,showpacs,superscriptaddress,nofootinbib]{revtex4-1}  
                
\usepackage{graphicx}  
\usepackage{dcolumn}   
\usepackage{bm}        
\usepackage{amssymb}   
\usepackage{epsfig}  
\usepackage{latexsym}
\usepackage{amssymb} 
\usepackage[english]{babel}
\usepackage{amsmath}
\usepackage{enumerate}
  
\newcommand{\apjl}{Astrophys. J. Lett.}
\newcommand{\aap}{Astron. Astrophys.}
\newcommand{\apjs}{Astrophys. J. Suppl. Ser.} 
\newcommand{\sa}{Sov. Astron.}
\newcommand{\jpb}{J. Phys. B.}
\newcommand{\natu}{Nature}
\newcommand{\aaps}{Astron. Astrophys. Supp. Ser.}
\newcommand{\aj}{Astron. J.}
\newcommand{\aas}{Bull. Am. Astron. Soc.}
\newcommand{\mnras}{Mon. Not. R. Astron. Soc.}
\newcommand{\pasp}{Publ. Astron. Soc. Pac.}
\newcommand{\memras}{Mem. R. Astr. Soc. }
\newcommand{\physrep}{Phys. Rep.}
\newcommand{\araa}{Annual Rev. Astron. Astrophys.}
\newcommand{\jcap}{Journal of Cosmology and Astroparticle Physics}
\newcommand{\ssr}{Space Science Reviews}
\newcommand{\aapr}{Astron. and Astrophys. Review}
\newcommand{\solphys}{Solar Physics}
\newcommand{\jqsrt}{J. Quant. Spec. Rad. Trans.}
    
\newcommand{\beq}{\begin{equation}}
\newcommand{\eeq}{\end{equation}}
\newcommand{\bga}{\begin{gathered}}
\newcommand{\ega}{\end{gathered}}
\newcommand{\eq}[1]{{Eq.~(#1)}}
\hyphenation{ALPGEN}
\hyphenation{EVTGEN}
\hyphenation{PYTHIA}


\begin{document}

\widetext    
\title{A new probe of magnetic fields in the pre--reionization epoch: II. Detectability}
\author{Vera Gluscevic}
\author{Tejaswi Venumadhav}
\affiliation{Institute for Advanced Study, Einstein Drive, Princeton, NJ 08540, USA} 
\author{Xiao Fang}
\author{Christopher Hirata}
\affiliation{Center for Cosmology and Astroparticle Physics, The Ohio State University, 191 West Woodruff Lane, Columbus, Ohio 43210, USA}
\author{Antonija Oklop\v ci\' c}
\author{Abhilash Mishra} 
\affiliation{California Institute of Technology, Mail Code 350-17, Pasadena, CA 91125, USA}
\date{\today}  
    
 
\begin{abstract} 
In the first paper of this series, we proposed a novel method to probe large--scale intergalactic magnetic fields during the cosmic Dark Ages, using 21--cm tomography. This method relies on the effect of spin alignment of hydrogen atoms in a cosmological setting, and on the effect of magnetic precession of the atoms on the statistics of the 21--cm brightness--temperature fluctuations. In this paper, we forecast the sensitivity of future tomographic surveys to detecting magnetic fields using this method. For this purpose, we develop a minimum--variance estimator formalism to capture the characteristic anisotropy signal using the two--point statistics of the brightness--temperature fluctuations. We find that, depending on the reionization history, and subject to the control of systematics from foreground subtraction, an array of dipole antennas in a compact--grid configuration with a collecting area slightly exceeding one square kilometer can achieve a detection threshold of $\sim 10^{-21}$ Gauss comoving (scaled to present--day value) within three years of observation. Using this method, tomographic 21--cm surveys could thus probe ten orders of magnitude below the current constraint on primordial magnetic fields from the CMB observations and provide exquisite sensitivity to magnetic fields \textit{in situ} at high redshift. 
\end{abstract} 
        
\pacs{} 
\maketitle  
\input introduction.tex
\input method_summary.tex
\vspace{-12pt}
\input basics.tex
\input estimator.tex
\input fisher.tex
\input results.tex
\input conclusions.tex
\input acknowledgements.tex  
\appendix 
\input appendix_Vrms.tex
\label{app:Vrms} 
\input appendix_lensing.tex
\label{app:lensing}
\vspace{-10pt}
\input appendix_fesc.tex
\label{app:fesc}

\bibliographystyle{apsrev4-1}
\bibliography{detectability}

\end{document}
