\section{Estimating the escape fraction of ionizing photons}

This Appendix describes our method for estimating the escape fraction of ionizing photons in semi--numerical simulations of the high--redshift 21--cm signal. We use this estimate to perform a sanity check of the fiducial model of the Lyman--$\alpha$ flux evolution (shown in Fig.~\ref{fig:cosmo}) used for the sensitivity calculations shown in \S\ref{sec:results}. We computed this model using \texttt{21CMFAST}. In order to match the calculations of Ref.~\cite{2012ApJ...746..125H} at the lower end of the relevant redshift range ($z\sim 15$), we changed two of the default input parameters, setting the star--formation efficiency to $0.0075$, and the population of ionizing sources to Population III stars. We then checked that these parameters satisfy the constraint that the escape fraction of ionizing photons is bound to be less than one, at all redshifts of interest. 

\texttt{21CMFAST} sidesteps the computationally expensive tasks of tracking individual radiation sources and performing the radiative transfer of ionizing photons (needed to simulate HII regions in the early universe). It uses an approximate relation between the statistics of HII regions and those of collapsed structures, the latter of which can be efficiently computed in pure large--scale--structure simulations \cite{2004ApJ...613....1F}. Thus, the escape fraction of ionizing photons is not a direct input to these simulations, but can be estimated using the procedure we describe below. 

The number of ionizing photons emitted in a given ionized region, integrated up to a fixed redshift, should equal the number of absorbed ionized photons. These read, respectively, 
\beq    
\bga
N_{\rm em} = \langle f_{\rm esc} \rangle f_\ast N_{\gamma/{\rm b}} f_{\rm coll} N_{\rm b}\\
N_{\rm abs} = f_{\rm H} (1 + \langle n_{\rm rec} \rangle) N_{\rm b},
\ega
\label{eq:pbalance}
\eeq 
where $f_{\rm H} = 0.924$ is the hydrogen number fraction; $f_\ast$ is the star--formation efficiency (the fraction of galactic baryonic mass in stars; this is an input parameter to \texttt{21CMFAST}); $N_{\gamma/{\rm b}}$ is the number of ionizing photons produced by stars per nucleus; $N_{\rm b}$ is the total number of nuclei within a given ionized region; $\langle f_{\rm esc} \rangle$ is the average escape fraction associated with a given region; $\langle n_{\rm rec} \rangle$ is the average number of recombinations per hydrogen atom inside that region; and $f_{\rm coll}$ is the collapse fraction therein. We assume that once regions are ionized, they stay ionized, and we also verified that the number of recombinations outside the ionized regions is negligible. 

Integrating the number of absorbed photons of Eq.~(\ref{eq:pbalance}) over the set $\mathcal{R}(z)$ of all ionized regions at a given redshift, we get the total number of absorbed ionizing photons,
\beq
\bga
N_{\rm abs, tot}(z) = f_{\rm H} \int_{\mathcal{R}(z)} n_{\rm b} dV  \\
  + f_{\rm H}^2 \int_z^\infty dz^\prime \biggr\vert \frac{dt}{dz^\prime} \biggr\vert \int_{\mathcal{R}(z^\prime)} \mathcal{C} n_{\rm b}^2 \alpha_{\rm B} \ dV , 
\ega
\label{eq:netabs}
\eeq
where $n_{\rm b}$ is the baryon number density; the Jacobian $|dt/dz|$ maps between redshift and proper time; $\mathcal{C} \equiv \langle n_\text{b}^2 \rangle/\langle n_\text{b} \rangle^2$ is the clumping factor; and $\alpha_{\rm B}$ is the case--B recombination coefficient (varies from ionized region to ionized region). On the other hand, using the \texttt{21CMFAST} ansatz that $f_{\rm coll} = 1/\zeta$, where $\zeta$ is an efficiency factor (also given as an input to the code), the total number of emitted ionizing photons reads
\beq
\bga
N_{\rm em, tot}(z)  = \frac{ \overline{f_{\rm esc}}(z) f_\ast N_{\gamma/{\rm b}} }{\zeta} \int_{\mathcal{R}(z)}  n_{\rm b} \ dV,
\ega 
\label{eq:netem}
\eeq
where $\overline{f_{\rm esc}}(z)$ is the overall averaged escape fraction up to redshift $z$---the quantity we aim to estimate. Combining Eqs.~(\ref{eq:netabs}) and (\ref{eq:netem}), we get
\beq
\bga
  \overline{f_{\rm esc}}(z) = \frac{f_{\rm H} \zeta}{ f_\ast N_{\gamma/{\rm b} } } \\
\times\left[ 1 + f_{\rm H} \frac{ \int_z^\infty dz^\prime \biggr\vert \frac{dt}{dz^\prime} \biggr\vert \int_{\mathcal{R}(z^\prime)} \mathcal{C} n_{\rm b}^2 \alpha_{\rm B} \ dV  }{ \int_{\mathcal{R}(z)} n_{\rm b} \ dV} \right].
\ega
\eeq
Rewriting the above integrals in terms of comoving coordinates $\vec r$ and the overdensity $\delta(\vec r, z)$, we finally get
\vspace{-5pt}
\beq
\bga
\overline{f_{\rm esc}}(z) = \frac{f_{\rm H} \zeta}{ f_\ast N_{\gamma/{\rm b} } } \\
\times  \left[ 1 + \frac{ f_{\rm H} n_{\rm b, \text{today}} }{ \int_{ \mathcal{R}(z)} d\vec r[1 + \delta(\vec r, z)] } \int_z^\infty dz^\prime \biggr\vert \frac{dt}{dz^\prime} \biggr\vert \right.\\
\times \left. (1 + z^\prime)^3 \int_{\mathcal{R}(z^\prime)} d\vec r \ \mathcal{C} [1 + \delta(\vec r, z^\prime)]^2 \alpha_{\rm B} \right] .
\ega
\eeq 
where $n_{\rm b, \text{today}}$ is the number density of baryons today. An additional subtlety is that \texttt{21CMFAST} follows the kinetic temperature in the IGM outside the ionized regions, while the recombination coefficient $\alpha_{\rm B}$ depends on the temperature inside these regions. In general, the latter differs from the former due to the energy deposited by the free--electrons released during photoionization. We simplify our calculation by setting the temperature inside the bubbles to $10^4$ K (corresponding to the mean kinetic energy of the particles of a few eV). 
