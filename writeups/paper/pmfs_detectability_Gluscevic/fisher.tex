\section{Fisher analysis}
\label{sec:fisher}

We now use the key results of \S\ref{sec:estimators} to evaluate sensitivity of future tomographic 21--cm surveys to detecting presence of magnetic fields in high--redshift IGM. In \S\ref{subsec:uniform_fisher}, we derive the expression for sensitivity to a field uniform in the entire survey volume. We start with the unsaturated case where (in the classical picture) hydrogen atoms complete less than a radian of magnetic precession during their lifetime in the triplet state, for all redshifts of interest (weak--field limit), and then move on to considering the saturated case (the fast--precession and stong--field limit). In \S\ref{subsec:SI_fisher}, we derive the expression for sensitivity to detecting a stochastic magnetic field described by a scale--invariant power spectrum.

%%%%%%%%%%%%%%%%%%%%%%%%%%%%%%%%%%%%%%%
\subsection{Uniform field}
\label{subsec:uniform_fisher}

Eq.~(\ref{eq:fisher_patch}) provides an expression for evaluating $1\sigma$ sensitivity to reconstructing a uniform magnetic field from measurements of the 21--cm signal at range of redshifts. For our numerical calculations, we take the following integration limits: $\phi_k\in[0,2\pi]$; $\theta_k\in [0,\pi]$; and $k\in[2\pi u_\mathrm{min}/(\chi(z)\sin\theta_k),2\pi u_\mathrm{max}/(\chi(z)\sin\theta_k)]$, where $u_\mathrm{min, max}=\frac{L_\text{min, max}}{\lambda}$ correspond to the maximum and minimum baseline lengths, $L_\text{min}$ and $L_\text{max}$, respectively. If the survey area is big enough that the flat--sky approximation breaks down, $\sigma_{B_0, \text{tot}}^{-2} $ can be evaluated on a small (approximately flat) patch of  size $\Omega_\text{patch}$ centered on the line of sight, and then corrected to account for the total survey volume
\footnote{This accounts for the change in the angle that a uniform magnetic field makes with a line of sight, as the line of sight ``scans'' through the survey area.} as
\beq
\bga
\sigma^{-2}_{ B_0,\text{corr}} = \frac{\sigma^{-2}_{ B_0,\text{tot}}}{\Omega_\text{patch}} \int_0^{\theta_\text{survey}}\int_{0}^{2\pi} \cos^2 \theta d\theta d\phi \\
= \frac{\pi\sigma^{-2}_{ B_0,\text{tot}}}{\Omega_\text{patch}} \left(\theta_\text{survey} + \cos \theta_\text{survey} \sin \theta_\text{survey}\right).
\ega
\label{eq:sigma_sum_survey}
\eeq

So far, we have only focused on the regime of a weak magnetic field. Let us now consider the case where the field is strong enough that the precession period is comparable to (or shorter than) the lifetime of the excited state of the forbidden transition---the saturated regime. In this case, the brightness--temperature signal still captures the presence of the field (as illustrated in Fig.~\ref{fig:hp}), but it loses information about the magnitude of the field, and can only be used to determine the lower limit of the field strength. The ability to distinguish the saturated case from zero magnetic field becomes a relevant measure of survey sensitivity in this scenario. 

To quantify the distinguishability of the two regimes, we write the signal power spectrum as the sum of contributions from both $B_0=0$ and $B_0\to\infty$, 
\beq
P^S(\vec k) = (1-\xi)P^S(\vec k, B=0) + \xi P^S(\vec k, B\to \infty).
\label{eq:saturated_P}
\eeq
We then perform the standard Fisher analysis to evaluate sensitivity to recovering parameter $\xi$,
\beq
\bga
\sigma_{\xi}^{-2} = 
\int dV_\mathrm{}(z)
\frac{d\vec k}{(2\pi)^3}\left(  \frac{\frac{\partial P^S}{\partial \xi}(\vec k)}{P^N (\vec k)+ P_0^S(\vec k,\xi=0) }\right)^2,
\ega
\label{eq:sigma_xi}
\eeq
where
\beq
\frac{\partial P^S}{\partial \xi}(\vec k) = P^S(\vec k, B\to \infty)-P^S(\vec k, B=0),
\eeq
and evaluating $P^S(\vec k, B\to \infty)$ requires the following limit of the transfer function (derived from Eq.~(\ref{eq:G_def}))
\beq
\bga
G({\bf{\widehat k}}, B\to \infty)
=\left( 1 - \frac{T_\gamma}{T_{\rm s}} \right) x_{1{\rm s}} \left( \frac{1+z}{10} \right)^{1/2} \\
\times \biggl[ 26.4 \ {\rm mK}  \left(1 + ({\bf{\widehat k}} \cdot {\bf{\widehat n}})^2 \right)  
- 0.128 \ {\rm mK} \left( \frac{T_\gamma}{T_{\rm s}} \right)\\
\times x_{1{\rm s}} \left( \frac{1+z}{10} \right)^{1/2}  
 \Bigl\{ 2 + 2({\bf{\widehat k}} \cdot {\bf{\widehat n}})^2 
- \frac{1}{60} \frac{1-3\cos ^2\theta_k}{1+x_{\alpha,(2)}+x_{c,(2)} }\Bigr\} \biggr].
\label{eq:G_Binf}
\ega
\eeq
Note that the above Equation is valid in the reference frame where the magnetic field is along the $z$ axis, and the line--of--sight direction is perpendicular to it. When evaluating Eq.~(\ref{eq:sigma_xi}) in \S\ref{sec:results}, we will only be interested in this configuration, since we aim to evaluate the sensitivity to the plane--of--the--sky component of $\vec B$. We interpret $\sigma_\xi^{-1}$ as 1$\sigma$ sensitivity to \textit{detecting} the presence of a strong magnetic field. 

%%%%%%%%%%%%%%%%%%%%%%%%%%%%%%%%%%%%%%%
\subsection{Stochastic field}
\label{subsec:SI_fisher}

To compute signal--to--noise ratio (SNR) for measuring the amplitude of a stochastic--field power spectrum, at a given redshift, we start with the general expression  
\beq
\text{SNR}^2 = \frac{1}{2} \text{Tr} \left( N^{-1}SN^{-1}S\right),
\label{eq:snr_general}
\eeq
where Tr denotes a trace of a matrix, and $S$ and $N$ stand for the signal and noise matrices, respectively. In the case of interest, these are $3N_\text{voxels}\times 3N_\text{voxels}$ matrices (there are 3 components of the magnetic field and $N_\text{voxels}$ voxels in the survey). In the null case, voxels are independent and the noise matrix is diagonal. Voxel--noise  variance for measuring a single mode is given by $P^N_{B_{0,i}}(\vec K, z)/V_\text{voxel} (z)$, where $V_\text{voxel}$ is voxel volume. Summing over all voxels and components of the magnetic field with inverse--variance weights gives
\beq
\bga
\text{SNR}^2 (z)= \frac{1}{2} \sum_{i\alpha, j\beta} \frac{S_{i\alpha , j\beta}^2}{P^N_{B_{0,i}}(\vec K, z)P^N_{B_{0,j}}(\vec K, z)} V_\text{voxel}^2\\=
\frac{1}{2} \sum_{ij} \int d\vec r_\alpha \int d\vec r_\beta \frac{\left< B_{0,i}(\vec r_\alpha) B_{0,j}(\vec r_\beta)\right>^2}{P^N_{B_{0,i}}(\vec K, z)P^N_{B_{0,j}}(\vec K, z)},
\ega
\label{eq:snr_z_step1}
\eeq
at a given redshift, where the Greek indices label individual voxels and, as before, Roman indices denote field components; $\vec r_{\alpha/\beta}$ represents spatial position of a given voxel. 

To simplify further calculations, we now focus on a particular class of magnetic--field models where most of the power is on largest scales (small $\vec K$). In this (squeezed) limit, $\vec K \ll \vec k$ and thus $\vec k \approx \vec k'$, such that \eq{\ref{eq:NK2}} reduces to white noise (independent of $\vec K$). A model for the power spectrum is defined through
\beq
(2\pi)^3\delta_D(\vec K - \vec K') P_{B_{0,i}B_{0,j}}(\vec K) \equiv \left<B_{0,i}^*(\vec K) B_{0,j}(\vec K')\right>,
\label{eq:Pbb}
\eeq
which relates to the variance in the transverse component $P_{B_\bot}(\vec K)$ as
\beq
P_{B_{0,i}B_{0,j}}(\vec K) = (\delta_{ij} - \widehat K_i \widehat K_j) P_{B_\bot}(\vec K),
\label{eq:Pbb_Pb}
\eeq
where $\widehat K_{i/j}$ is a unit vector along the direction of the ${i/j}$ component of the wavevector.
In the rest of this discussion, for concreteness, we consider a scale--invariant (SI) power spectrum, 
\beq
P_{{B_\bot}}(\vec K) = A_0^2/K^3.
\label{eq:SI}
\eeq
Here, the amplitude $A_0$ is a free parameter of the model (in units of Gauss).

If homogeneity and isotropy are satisfied, the integrand in Eq.~(\ref{eq:snr_z_step1}) only depends on the separation vector $\vec s \equiv \vec r_\beta -\vec r_\alpha$. Using this and the squeezed limit assumption gives\footnote{In the last step, we used $\int d\vec s |f(\vec s)|^2 = \int \frac{d\vec K}{(2\pi)^3}|\widetilde f(\vec K)|^2$, which holds for an arbitrary function $f$ and its Fourier transform $\widetilde f$.}
\beq  
\bga
\text{SNR}^2 (z) = 
\frac{1}{2} \sum_{ij}  \frac{dV_\text{patch}}{{(P^N_{B_{0,i}}(z))^2}}\int d\vec s \left< B_{0,}i(\vec r_\beta - \vec s) B_{0,j}(\vec r_\beta)\right>^2
\\=
\frac{1}{2(2\pi)^3} \sum_{ij}  \frac{dV_\text{patch}}{{(P^N_{B_{0,i}}(z))^2}} \int d\vec K\left(P_{B_{0,i}B_{0,j}}(\vec K)\right)^2,
\ega
\label{eq:snr_z}
\eeq
where $dV_\text{patch}$ is the volume of a redshift--slice patch defined in \eq{\ref{eq:dVpatch}}. After substituting \eq{\ref{eq:SI}} and integrating over redshifts the total SNR is given by
\beq
\bga
\text{SNR}^2 =  \frac{A_0^4}{2(2\pi)^3}  \int_{z_\text{min}}^{z_\text{max}}\frac{dV_\text{patch}}{{(P^N_{B_{0,i}}(z))^2}}
\int_0^{\pi} \sin\theta d\theta \\
\int_0^{2\pi} d\phi\int_{K_\text{min}(z,\theta,\phi)}^{K_\text{max}(z,\theta,\phi)} \frac{d K}{K^4}\sum_{ij\in \{xx, xy, yx, yy\}}(\delta_{ij} - \widehat K_i\widehat K_j)^2,
\ega
\label{eq:snr_intK}
\eeq
where $x$ and $y$ denote components in the plane of the sky, and
\beq
\widehat K_x = \sin\theta\sin\phi, \text{     }
\widehat K_y = \sin\theta\cos\phi.
\label{eq:hat_K_xy}
\eeq
The sum in the above expression reduces to
\beq
\sum_{ij\in \{xx, xy, yx, yy\}}(\delta_{ij} - \widehat K_i\widehat K_j)^2 = 2\cos^2\theta+\sin^4\theta.
\label{eq:sumij}
\eeq
Substituting  Eq.~(\ref{eq:sumij}) into Eq.~(\ref{eq:snr_intK}) and integrating over $K,\theta,\phi$ gives
\beq
\text{SNR}^2 =  \frac{A_0^4}{10\pi^2} \int_{z_\text{min}}^{z_\text{max}}\frac{dV_\text{patch}}{{(P^N_{B_{0,i}}(z))^2}} \left(\frac{1}{K_\text{min}^3}-\frac{1}{K_\text{max}^3}\right).
\label{eq:snr_ints}
\eeq
Finally, from the above expression, $1\sigma$ sensitivity to measuring $A_0^2$ is given by
\beq
\sigma^{-2}_{A_0^2} =  \frac{1}{10\pi^2} \int_{z_\text{min}}^{z_\text{max}}\frac{dV_\text{patch}}{{(P^N_{B_{0,i}}(z))^2}} \left(\frac{1}{K_\text{min}^3}-\frac{1}{K_\text{max}^3}\right).
\label{eq:sigma_A0}
\eeq
Note at the end that, for our choice of the SI power spectrum, the choice of $K_\text{max}$ does not matter (contribution to sensitivity rapidly decreases at small scales), while we take $K_\text{min}=2\pi /(\chi(z)\sin\theta_k)$ to match the survey size at a given redshift, for the compact--array configuration considered throughout this work. 