\section{Introduction}
\label{sec:intro}

Magnetic fields are ubiquitous in the universe on all observed scales \cite{2013A&ARv..21...62D,Vallee04,Neronov10,2005LNP...664...89W,2012SSRv..166..215B}. However, the origins of the magnetic fields in Galaxies and on large scales are as of yet an unresolved question. Various forms of dynamo mechanism have been proposed to maintain and amplify magnetic fields \cite{2013PhRvE..87e3110P}, but they typically require the presence of seed fields \cite{2013A&ARv..21...62D}. Such seed fields may be produced during structure formation through the Biermann battery process or similar mechanisms \cite{Naoz13,2013PhRvL.111e1303N}, or may otherwise be relics from the early universe \cite{2013A&ARv..21...62D,2012SSRv..166...37W,2014JCAP...05..040K}. Observations of large--scale low--strength magnetic fields in the high--redshift intergalactic medium (IGM) could thus probe the origins of present--day magnetic fields and potentially open up an entirely new window into the physics of the early universe.

Many observational probes have been previously proposed and used to search for evidence of large--scale magnetic fields locally and at high redshifts (e.~g.~\cite{Yamazaki10,Blasi99,Tavecchio10,Dolag11,2005LNP...664...89W,2014JCAP...01..009K,2013ApJ...770...47K,2014PhRvD..89j3522S,2006MNRAS.372.1060T,2009ApJ...692..236S}). Amongst the most sensitive tracers of cosmological magnetic fields is the cumulative effect of Faraday rotation in the cosmic-microwave-background (CMB) polarization maps, which currently places an upper limit of $\sim$$10^{-10}$ Gauss (in comoving units) using data from the Planck satellite \cite{2015arXiv150201594P}. In Paper I of this series \cite{2014arXiv1410.2250V}, we proposed a novel method to detect and measure extremely weak cosmological magnetic fields during the pre-reionization epoch (the cosmological Dark Ages). This method relies on data from upcoming and future 21--cm tomography surveys \cite{1997ApJ...475..429M,2004PhRvL..92u1301L}, many of which have pathfinder experiments currently running \cite{2012arXiv1201.1700G,2011AAS...21813206B,2014ApJ...788..106P,2008arXiv0802.1727C,Vanderlinde14,2015AAS...22532803D}, with the next-stage experiments planned for the coming decade \cite{2008arXiv0802.1727C,2015AAS...22532803D}. 

In Paper I, we calculted the effect of a magnetic fields on the statistics of the 21--cm signal, and in this paper (which we refer to as Paper II in the following), we focus on evaluating the sensitivity of future 21--cm experiments to this effect. As we discussed in Paper I, measurement of statistical anisotropy in the 21--cm signal from the Dark Ages has an intrinsic sensitivity to magnetic fields in the IGM more than \textit{ten orders of magnitude below the current upper limits from the CMB}. In the following, we demonstrate that a square--kilometer array of dipole antennas in a compact grid can reach the sensitivity necessary to detect large--scale magnetic fields that are on the order of $10^{-21}$ Gauss comoving (scaled to present day, assuming adiabatic evolution of the field due to Hubble expansion). 

The rest of this paper is organized as follows. In \S\ref{sec:method}, we summarize the main results of Paper I. In \S\ref{sec:basics}, we define our notation and review the basics of the 21--cm signal and its measurement. In \S\ref{sec:estimators}, we derive minimum--variance estimators for uniform and stochastic magnetic fields. In \S\ref{sec:fisher}, we set up the Fisher analysis formalism necessary to evaluate detectability. In \S\ref{sec:results}, we present numerical results, and we conclude in \S\ref{sec:conclusions}. Supporting materials are presented in the appendices.