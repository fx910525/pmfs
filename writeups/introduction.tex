\section{Introduction}
\label{sec:intro}

Magnetic fields are ubiquitous in the universe on all observed scales \cite{2013A&ARv..21...62D,Vallee04,Neronov10,2005LNP...664...89W,2012SSRv..166..215B}. However, the origins of the magnetic fields in Galaxies and on large scales are as of yet an unresolved question. Various forms of dynamo mechanisms \cite{2013PhRvE..87e3110P} are proposed to maintain and amplify magnetic fields, but they typically require seed fields to act \cite{2013A&ARv..21...62D}. The seed fields may be produced during structure formation through Biermann battery or similar mechanisms \cite{Naoz13,2013PhRvL.111e1303N}, or otherwise may be relics from the early universe \cite{2013A&ARv..21...62D,2012SSRv..166...37W,2014JCAP...05..040K}. Observations of large-scale weak magnetic fields in the high-redshift intergalactic medium (IGM) can thus provide tools for understanding the origins of magnetic fields in the present-day universe, and potentially open up an entirely new window into the physics of the early universe.

Many observational probes have previously been used to search for evidence of large-scale magnetic fields locally and at high redshifts; see, e.~g.~\cite{Yamazaki10,Blasi99,Tavecchio10,Dolag11,2005LNP...664...89W,2014JCAP...01..009K,2013ApJ...770...47K,2014PhRvD..89j3522S,2006MNRAS.372.1060T,2009ApJ...692..236S}. Amongst the most sensitive tracers of cosmological magnetic fields are the measurements of the cumulative effect of Faraday rotation in the cosmic-microwave-background (CMB) polarization maps, which currently place an upper limit of $\sim$$10^{-10}$ Gauss (in comoving units) using the data from the Planck sattelite \cite{2015arXiv150201594P}. In Paper I of this series \cite{2014arXiv1410.2250V}, we proposed a novel method to detect and measure extremely weak cosmological magnetic fields during the pre-reionization epoch (the Dark Ages). This method relies on future 21-cm brightness-temperature tomography surveys \cite{1997ApJ...475..429M,2004PhRvL..92u1301L}, many of which have pathfinder experiments currently running \cite{2012arXiv1201.1700G,2011AAS...21813206B,2014ApJ...788..106P,2008arXiv0802.1727C,Vanderlinde14,2015AAS...22532803D}, and plans for the next stages to be realized in the coming decade \cite{2008arXiv0802.1727C,2015AAS...22532803D}. As we show in Paper I, the measurement of statistical anisotropy in the 21-cm signal from the Dark Ages has intrinsic sensitivity to magentic fields in the IGM more than \textit{10 orders of magnitude below the current upper limits}. 

While Paper I layed out the formalism necessary to account for the effect of  magnetic fields on the statistics of the 21-cm signal, this paper (which we refer to as Paper II in the following) focuses on evaluating the sensitivity of future 21-cm experiments using this method. The rest of this paper is organized as follows. In \S\ref{sec:method}, we present a quick overview of the main results in Paper I. In \S\ref{sec:estimators}, we derive minimum-variance estimators for a uniform and stochastic magnetic field. In \S\ref{sec:fisher}, we set up the Fisher analysis formalism necessary to evaluate detectability. In \S\ref{sec:results}, we present numerical results, and we conclude in \S\ref{sec:conclusions}. Supporting materials are presented in the appendicies.