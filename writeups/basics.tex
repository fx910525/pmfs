\section{Basics}
\label{sec:basics}
\subsection{Definitions}
\label{subsec:def}

The redshifted 21-cm signal can be represented as a specific intensity at  the location in physical space $I(\vec{r})$, or in Fourier space $\widetilde{I}(\vec{k})$. In sky coordinates, these functions become $\mathcal{I}(\theta_x, \theta_y, \theta_\nu)$, and $\widetilde{\mathcal{I}}(u,v,\eta)$, respectively. Here, vector $\vec{k}$ (in the units of comoving Mpc$^{-1}$) is a Fourier dual of $\vec{r}$ (comoving Mpc), and likewise, $\theta_x$ (rad), $\theta_y$ (rad), and $\theta_\nu$ (Hz) are duals of the coordinates in the $uv$-plane $u$ (rad$^{-1}$), $v$ (rad$^{-1}$), and $\eta$ (seconds), respectively. These two sets of coordinates are related through linear transformations in the following way
\begin{equation}
\begin{gathered}
\theta_x = \frac{r_x}{D_M(z)}, \hspace{0.5in} u = \frac{k_xD_M(z)}{2\pi},\\
\theta_y = \frac{r_y}{D_M(z)}, \hspace{0.5in} v = \frac{k_yD_M(z)}{2\pi},\\
\theta_\nu = \frac{H(z)\nu_{21}}{c(1+z)^2} r_z, \hspace{0.5in} \eta = \frac{c(1+z)^2}{2\pi H(z)\nu_{21}}k_z,
\end{gathered}
\label{eq:fourier_duals}
\end{equation} 
where $\nu_{21}$ is the 21-cm frequency in the rest frame, $H(z)$ iz the Hubble parameter, $D_M(z)$ is the comoving distance in transverse direction, and z is the reference redshift in the middle of the observed data cube (where $r_z$ and $\theta_\nu$ intervals are evaluated). Note that conditions of the type $2\pi\theta_xu = r_xk_x$ are saticefied.

The convention we use for the Fourier transforms is (note that Fourier-space functions are denoted with tilde)
\beq
\bga
I(\vec{r}) = \int\widetilde{I}(\vec{k})e^{i\vec{k} \cdot \vec{r}}d^3\vec{k},\\
\widetilde{I}(\vec{k}) = \frac{1}{(2\pi)^3}\int{I}(\vec{r})e^{-i\vec{k} \cdot \vec{r}}d^3\vec{r},
\ega
\label{eq:tildeI_I}
\eeq
and similarly,
\beq
\bga
\mathcal{I}(\theta_x,\theta_y,\theta_\nu) = \int\widetilde{\mathcal{I}}(u,v,\eta)e^{2\pi i(u\theta_x + v\theta_y+\eta \theta_\nu)}dudvd\eta,\\
\widetilde{\mathcal{I}}(u,v,\eta) = \int{\mathcal{I}}(\theta_x,\theta_y,\theta_\nu)e^{-2\pi i(u\theta_x + v\theta_y+\eta\theta_\nu)}d\theta_xd\theta_yd\theta_\nu.
\ega
\label{eq:mathcal_tildeI_I}
\eeq
The following scaling relation is saticefied
\beq
\widetilde{I}(\vec{k}) = \frac{c(1+z)^2D_M(z)^2}{(2\pi)^3H(z)\nu_{21}}\widetilde{\mathcal{I}}(u,v,\eta),
\label{eq_tilde_I_vs_Ik_scaling}
\eeq
where the proportionallity factor is a Jacobian $\frac{d\theta_xd\theta_yd\theta_\nu}{dr_xdr_ydr_z}$. 

Finally, the relationship between the specific intensity in the $uv$-plane and the visibility function $V(u,v,\theta_\nu)$ is given by a Fourier transform over only the frequency coordinate,
\beq
\bga
 V(u,v,\theta_\nu)= \int \mathcal{\widetilde{I}}(u,v,\eta)e^{2\pi i \theta_\nu\eta}d\eta,\\
\mathcal{\widetilde{I}}(u,v,\eta) = \int V(u,v,\theta_\nu)e^{-2\pi i \theta_\nu\eta}d\theta_\nu,
\ega
\label{eq:visibility}
\eeq
Here, $\theta_\nu$ is a discrete variable such that $\theta_{\nu,\text{max}}-\theta_{\nu,\text{min}}=\Delta\nu$ is the bandwidth of the observed data cube centered on $z$ (see also Appendix \ref{app_Vrms}).
%%%%%%%%%%%%%%
\subsection{Power spectra and noise}
\label{subsec:noise}

In this Section, we derive the noise power spectrum for the brightness temperature signal. We start by defining a brightness-temperature power spectrum as
\beq
\langle \widetilde{I}(\vec{k})\widetilde{I}^*(\vec{k}')\rangle \equiv (2\pi)^3P_{\widetilde{I}}\delta(\vec{k}-\vec{k}').
\label{eq_tildeI_power}
\eeq
The observable we wish to relate this power spectrum to is the visibility function---a complex Gaussian variable with a zero mean, whose noise-induced variance (derived in Appendix \ref{}) reads
\beq\bga
\langle V(\vec{u},f)V(\vec{u}',f')^*\rangle \\
= \left(\frac{2k_BT_\text{sys}}{A_e\sqrt{\Delta\nu t_1}}\right)^2 \delta_D({u}-{u}')\delta_D({v}-{v}')\delta_{\theta_\nu\theta_{\nu}'},
\ega
\label{eq_Vrms}
\eeq 
where $T_\text{sys}$ is the system + sky temperature; $t_1$ is the total time a single baseline spent observing an element at the position $(u,v)$ in the $uv$ plane; $A_e$ is the collecting area of a single dish; $k_B$ is Boltzmann constant; $\Delta\nu$ is the bandwidth of the survey; and $\delta$ symbols denote the Dirac delta and the Kronecker delta, respectively.

The next step is to combine Eqs.~(\ref{eq:visibility}) and (\ref{eq_Vrms}), and take ensamble average to get
\beq
\bga
\langle\widetilde{\mathcal{I}}(u,v,\eta) \widetilde{\mathcal{I}}^*(u',v',\eta')\rangle \\
 = \frac{1}{t_1}\left(\frac{2k_BT_\text{sys}}{A_e}\right)^2 \delta(\vec{u}-\vec{u}')\delta(\eta-\eta'),
\ega
\label{eq:mathcal_power_Vrms}
\eeq 
where 
\beq
\int e^{2\pi i \theta_\nu(\eta-\eta')}d\theta_\nu =\delta(\eta-\eta'),
\eeq
is the periodic delta-function on the $t_1$ interval.  Taking into account the scaling relation of Eq.~(\ref{eq_tilde_I_vs_Ik_scaling}), introducing the power spectrum of Eq.~(\ref{eq_tildeI_power}), and keeping in mind the scaling property of the delta function, we arrive at
\beq
P_1^N(\vec k) = \frac{1}{t_1}\frac{c(1+z)^2D_M^2(z)}{H(z)\nu_{21}}\left(\frac{2k_BT_\text{sys}}{A_e}\right)^2 ,
\label{eq:Pnoise_1mode}
\eeq
for the noise per $\vec k$ mode, per baseline.

Computation of $t_1$ from the total duration of the survey  $t_\text{obs}$ depends on the type of the experiment.  For a beam of a solid angle $\Omega_\text{beam}=\lambda^2/A_e$ much smaller than the solid angle of the entire survey, $\Omega_\text{survey}$, where telescopes scan the sky one beamwidth at a time (such as the case for radio dishes), $t_1$ is the total time spent observing one $uv$ element of size corresponding to the beam size, $t_1=t_\text{obs}\Omega_\text{survey}/\Omega_\text{beam}$. However, in the case of an array of dipoles, the beam is greater or equal to the survey angular size, and $t_1=t_\text{obs}$. When deriving numerical results in \S\ref{sec:results}, we will assume the latter. Finally, we will not account for the fact that a given patch of the sky is only visible for a part of a day from a given location; therefore, $t_\text{obs}$ we use in \S\ref{sec:results} is the survey duration times a factor of a few or less that corrects for this assumption.

The last step is to get from Eq.~(\ref{eq:Pnoise_1mode}) to the expression for the noise power spectrum that corresponds to the observation with all the available baselines. To do that, we need to incorporate the knowledge about the array configuration and the coverage of the $uv$ plane. In other words, we need to divide the expression in Eq.~(\ref{eq:Pnoise_1mode}) by the number of baselines that see a given mode $\vec k$ at any given time $n_\text{base}(\vec k)$ (for a discussion of the $uv$ coverage, see the following section). The final result for the noise power spectrum per mode $\vec k$ in the intensity units is then
\beq
P^N(\vec k) = \frac{1}{t_\text{1}}\frac{c(1+z)^2D_M^2(z)}{H(z)\nu_{21}}\frac{\left(2k_BT_\text{sys}\right)^2}{A_e^2n_\text{base}(\vec k)},
\label{eq:Pnoise_Jy}
\eeq
and in temperature units
\beq
P^N(\vec k) = \frac{\lambda^4}{t_\text{1}}\frac{c(1+z)^2D_M^2(z)}{H(z)\nu_{21}}\frac{T_\text{sys}^2}{A_e^2n_\text{base}(\vec k)}.
\label{eq:Pnoise_K}
\eeq
%%%%%%%%%%%%
\subsection{The UV coverage}
\label{subsec:uv}

Total number density $n_\text{base}(\vec k)$ of baselines that can observe mode $\vec k$ is related to the (unitless) number density $n(u,v)$ of baselines per $dudv$ element as
\beq
n_\text{base}(\vec k) = \frac{n(u,v)}{\Omega_\text{beam}},
\label{eq:nuv_nk}
\eeq
where $\frac{1}{\Omega_\text{beam}}$ represents an element in the $uv$ plane. The number density integrates to the total number of baselines $N_\text{base}$,
\beq
N_\text{base}=\frac{1}{2}N_\text{ant}(N_\text{ant}+1) = \int_\text{half} n(u,v)dudv,
\label{eq:nk}
\eeq
where $N_\text{ant}$ is the number of antennas in the array, and the integration is done on the half of the $uv$ plane (because the visibility has the following property $V(u,v,\theta_\nu)=V^*(-u,-v,\theta_\nu)$, and only half the plane contains independent samples). We assume that the array consists of many antennas, so that time-dependence of $n(u,v)$ is negligible; if this is not the case, one should compute its time average to account for Earth's rotation.

Let us now consider $n_\text{base}(\vec k)$ for a specific array configuration that is of particualr interest to cosmology---a tightly packed array of simple dipole antennas, tiling a squared-surface of the area $(\Delta L)^2$ with a filling fraction close to one. This is a design such as the FFT telescope described in \cite{Tegmark and Zaldarriaga (2009)}. In this case, the beam solid angle is 1 sr, the affective area of a single dipole is $A_e = \lambda^2$, and the effective number of antennas is then $N_\text{ant} = \frac{(\Delta L)^2}{\lambda^2}$. For such configuration
\beq
n(u,v) = (\frac{\Delta L}{\lambda} - u)(\frac{\Delta L}{\lambda} - v).
\label{eq:nuv_fftt}
\eeq
The relation between $\vec k=(k,\theta_k,\phi_k)$ and $(u,v)$ is
\beq
\bga
u_\perp \equiv \frac{D_A(z)}{2\pi}k\sin\theta_k,\\
u = u_\perp \cos\phi_k,\\
v = u_\perp \sin\phi_k,
\ega
\label{eq:k_uv}
\eeq
where subscript $\perp$ denotes components perpendicular to the line-of-sight (LOS) direction, $\widehat{n}$, and $D_A$ is the angular diameter distance. From this, the corresponsing number of baselines observing a given $\vec k$ is
\beq
\bga
n_\text{base}(\vec k) = (\frac{\Delta L}{\lambda} - \frac{D_A(z)}{2\pi}k\sin\theta_k\cos\phi_k)\\\times (\frac{\Delta L}{\lambda} - \frac{D_A(z)}{2\pi}k\sin\theta_k\sin\phi_k).
\ega
\label{eq:nk_fftt}
\eeq
In \S\ref{sec:results}, we use a $\phi_k$-averaged version of this quantity (between $0$ and $\pi/2$ only, due to the four-fold symmetry of the experimental setup of a square of dipoles), to account for the rotation of the baselines with respect to the modes,
\beq
\bga
\langle n_\text{base}(\vec k) \rangle_{\phi_k}= \left(\frac{\Delta L}{\lambda}\right)^2 -\frac{4}{\pi}\frac{\Delta L}{\lambda}\frac{D_A(z)}{2\pi}k\sin\theta_k \\+ \left(\frac{D_A(z)}{2\pi}k\sin\theta_k\right)^2.
\ega
\label{eq:nk_fftt_mean}
\eeq