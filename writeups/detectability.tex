\documentclass[aps,prd,twocolumn,floatfix,showpacs,superscriptaddress,nofootinbib]{revtex4-1}  
     
\usepackage{graphicx}  
\usepackage{dcolumn}   
\usepackage{bm}        
\usepackage{amssymb}   
\usepackage{epsfig}  
\usepackage{latexsym}
\usepackage{amssymb}
\usepackage[english]{babel}
\usepackage{amsmath}
\usepackage{enumerate}
  
\newcommand{\apjl}{Astrophys. J. Lett.}
\newcommand{\aap}{Astron. Astrophys.}
\newcommand{\apjs}{Astrophys. J. Suppl. Ser.}
\newcommand{\sa}{Sov. Astron.}
\newcommand{\jpb}{J. Phys. B.}
\newcommand{\natu}{Nature}
\newcommand{\aaps}{Astron. Astrophys. Supp. Ser.}
\newcommand{\aj}{Astron. J.}
\newcommand{\aas}{Bull. Am. Astron. Soc.}
\newcommand{\mnras}{Mon. Not. R. Astron. Soc.}
\newcommand{\pasp}{Publ. Astron. Soc. Pac.}
\newcommand{\memras}{Mem. R. Astr. Soc. }
\newcommand{\physrep}{Phys. Rep.}
\newcommand{\araa}{Annual Rev. Astron. Astrophys.}
\newcommand{\jcap}{Journal of Cosmology and Astroparticle Physics}
\newcommand{\ssr}{Space Science Reviews}
\newcommand{\aapr}{Astron. and Astrophys. Review}
\newcommand{\solphys}{Solar Physics}
\newcommand{\jqsrt}{J. Quant. Spec. Rad. Trans.}
   
\newcommand{\beq}{\begin{equation}}
\newcommand{\eeq}{\end{equation}}
\newcommand{\bga}{\begin{gathered}}
\newcommand{\ega}{\end{gathered}}
\newcommand{\eq}[1]{{Eq.~(#1)}}
\hyphenation{ALPGEN}
\hyphenation{EVTGEN}
\hyphenation{PYTHIA}


\begin{document}

\widetext
\title{A new probe of magnetic fields in the pre-reionization epoch: II. Detectability}
\author{Vera Gluscevic}
\author{Tejaswi Venumadhav}
\affiliation{Institute for Advanced Study, Einstein Drive, Princeton, NJ 08540, USA} 
\author{Abhilash Mishra} 
\author{Antonija Oklopcic}
\affiliation{California Institute of Technology, Mail Code 350-17, Pasadena, CA 91125, USA}
\author{Christopher Hirata}
\affiliation{Center for Cosmology and Astroparticle Physics, The Ohio State University, 191 West Woodruff Lane, Columbus, Ohio 43210, USA}
\date{\today}
 

\begin{abstract}
In the first paper of this series, we proposed a novel method to detect large-scale intergalactic magnetic fields during the Dark Ages, using future 21-cm tomography surveys. In this paper, we examine detectability of magnetic fields using this method. We first develop a minimum-variance estimator formalism that relies on identifying the characteristic anisotropic imprints in the 21-cm brightness-temperature 2-point correlation functions. We then perform Fisher forecast for this estimator and find that a radio array consiting of a square kilometer of dipole antennas (such as the next-generation of HERA, and the SKA) can detect fields of strength on the order of $10^{-21}$ Gauss at redshifts of 20, reaching almost 10 orders of magnitude below the current CMB constraints. 
\end{abstract} 
  
\pacs{} 
\maketitle
\input introduction.tex
\vspace{-15pt}
\input method_summary.tex
\input basics.tex
\input estimator.tex
\input fisher.tex
\input results.tex
\input conclusions.tex
\input acknowledgements.tex  
\appendix 
\input appendixA.tex
\label{app:Vrms}

\bibliographystyle{apsrev4-1}
\bibliography{detectability}

\end{document}
